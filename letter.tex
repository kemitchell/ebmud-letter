\documentclass[letterpaper]{letter}
\usepackage{fontspec}
\setromanfont[
	SmallCapsFont={Equity Caps A},
	UprightFont={Equity Text A},
	BoldFont={Equity Text A Bold},
	ItalicFont={Equity Text A Italic},
	BoldItalicFont={Equity Text A Bold Italic}]{Equity Text A}
\usepackage{polyglossia}
\setmainlanguage{english}
\usepackage{soul}
\date{November 27, 2020}
\begin{document}
  \begin{letter}{%
    East Bay Municipal Utility District\\
    Attention: Board of Directors\\
    375 11th Street\\
    Oakland, California 94607}

    \opening{EBMUD:}

    We are leaders of homeless camps, health professionals, social workers, charities, and volunteers in the East Bay. We write to share our experience of EBMUD’s impact on homeless people, as well as the impact of the homeless people on EBMUD. We also write to propose specific, practical opportunities for EBMUD to take a leadership role in addressing the water aspect of the housing crisis, while at the same time reducing costs and avoiding damage to our communities’ critical infrastructure.

    In short, we see that our homeless neighbors and EBMUD are often at odds. We strongly believe they needn’t be. We write to ask EBMUD’s help, and to offer our own, to make new ways for EBMUD and homeless people to work together.

    \begin{samepage}
    \textbf{Summary}

    People without permanent homes need clean, safe drinking water, just like their neighbors. But they cannot get EBMUD service. Without it, they pay dearly and daily for the water they need. Too often, they suffer without.

    Desperate people take desperate measures. Those measures have cost EBMUD and hurt our critical water infrastructure. EBMUD has responded, logically and understandably. The camps have responded in turn, just so. EBMUD can break that cycle and promote safe, responsible access with two new initiatives:

    \emph{Good Samaritan Rule}: a stopgap and bridge to longer-term solutions

    \begin{itemize}
      \item Change service rules to allow sharing with homeless people off-premises.
      \item Limit to a sane but meaningful gallon volume per month.
      \item Continue to prohibit resale of water.
    \end{itemize}

    \emph{Hydrant Access Program}: a sustainable, self-sufficient access option

    \begin{itemize}
      \item Give nonprofits access to hydrants to supply homeless people.
      \item Innovate on the ⅝-inch hydrant meter programs of neighboring utilities.
      \item Require use of rated, commercially available filters and water testing.
    \end{itemize}

    The following pages develop these proposals.
    \end{samepage}

    \stopbreaks
    \textbf{Reality}

    We know you are aware of the current and growing housing crisis in our communities. We know you are aware of the reasons to expect more people making do on our streets in months to come. At the risk of repeating some other things you already know, we’d like to report a bit of what we see “on the ground” where that crisis touches the basic human need for water.
    \startbreaks

    Many unhoused people in our community do not have access to EBMUD service, or to any regular, reliable, and safe source of water. The impact of this hardship on day-to-day life cannot be put into words, especially in warmer months, during heat waves, and when the air fills with smoke. Desperately thirsty people can’t be themselves. The deep body panic of thirst makes every part of a hard life harder. The immediate and long-term consequences, both physical and psychological, can be dire. And lasting.

    Water can be had from stores, not just from EBMUD. But the cost of water “off the shelf” amplifies the punishing cost of poverty. Out-the-door prices for bulk bottled water in Oakland hover near a dollar a gallon, before tax and transportation. That cost adds up fast.

    The United Nations High Commissioner for Refugees’ Water, Sanitation and Hygiene Manual specifies a minimum of twenty liters, or more than five gallons, of clean water per person per day at camp. For a week, that’s roughly forty gallons per person, or more than three hundred pounds by weight. In local practice, even small camps can easily empty a hundred gallons per week or more, especially in warm weather. A hundred gallons per week is a hundred dollars spent on being homeless, rather than on housing, health, or opportunity.

    This assumes store-bought water represents a viable option. Often it simply doesn’t.

    Cost aside, many homeless people don’t have cars, or don’t have cars that run. Over time, displacement tends to push them further from walkable commercial corridors and residential areas into industrial areas and so-called “food deserts” with no stores. They have further to go, and no way to get there.

    Even when stores are near, the elderly, disabled, injured, and sick can’t haul store-bought water in volume. Even the young and healthy often can’t move enough for their growing communities. If water doesn’t come to these people, one way or another, they suffer.

    Even assuming funds, health, time, and transportation for store-bought water, that water comes at considerable cost, and not just in dollars. Paper-thin plastic, single-use bottles and jugs quickly become unsightly, voluminous trash. When collection comes unreliably or infrequently, or when it never comes at all, bottles and jugs pile up, harming health, attracting pests, and creating fire and physical hazards. Of course, heaping piles of plastic also look terrible, to the unhoused and their neighbors alike. The complaints that result hasten the cycle of displacement into ever more remote “water deserts.”

    Several desperate camps have resorted to opening fire hydrants. This is clearly illegal. Legality aside, the water from hydrants often looks and smells bad, especially when those operating the hydrant haven’t been trained or equipped to flush or drain one properly. People are understandably reluctant to drink or cook with such water, despite their desperate need. So we have seen camps that access hydrants for washing often remain dependent on stores or charity for water to drink. Some camps boil iffy water, invariably over open flames, at great cost in fuel and fire risk. Meanwhile, clean, safe EBMUD drinking water often courses through pipes beneath their feet, to premises nearby.

    We also see that improper use of hydrants, without specialized tools, often results in stripped fasteners. This makes it difficult, slow, or impossible for firefighters to operate the hydrants when needed. Too often, the fire in need of the hydrant is a fire in a camp, started by an open flame. In industrial areas, where fires can be especially large and costly, every hydrant counts, and so does every second wasted struggling to open it.

    We also fear that improper operation of hydrants, like incomplete draining and fast or incomplete closing of valves, risks hydrant damage, hydraulic shock, and backflow contamination in the system. We understand that these issues threaten all those depending on our infrastructure, and not just people without homes.

    We suspect these concerns explain the appearance of security locks on hydrants near some camps. We understand the motivation for those locks. We also understand the motivation to remove them, and see that some camps have managed to do so. Others have responded to locked hydrants by tapping unlocked water pipes nearby. All of this is plainly illegal. Legality aside, it is also totally predictable.

    Some far-flung camps have relied on the generosity of those who do have regular water service, both residential and commercial, to avoid tampering with EBMUD infrastructure. Some are able to access water hoses or faucets at nearby houses or businesses, invited and not. Others rely on Good Samaritans to fill and transport water by car or truck. Sustainable or not, practical or not, people simply cannot choose to do without water. Necessity breeds invention, but not all inventions work out so well. Not for EBMUD, not for surrounding communities, and not for the inventors.

    Overall, we see that EBMUD has acted altogether understandably, from its vantage point, to curtail illegal activity, prevent damage to its infrastructure, and defend the safety of its water. We see that the people making do on our streets, likewise, have acted understandably from their own position, to meet the basic human need for water. We have not seen both sides acting together, from a broader, shared perspective. Not yet.

    \stopbreaks
    \textbf{Opportunity}

    We write to offer two concrete, actionable steps EBMUD can take to aid our unhoused neighbors and reduce harm and costs from water desperation: first, a Good Samaritan rule for sharing water with homeless people, and second, an extension to the hydrant meter program for nonprofit organizations serving homeless camps. We’re ready to work with EBMUD to bring these and other measures to bear, to end the escalating cat-and-mouse game on our streets, and to replace it with a constructive effort.
    \startbreaks

    In short: Responsible access to EBMUD water for unhoused people can be a safe, legitimate use of our public water system. A path toward legitimate, reliable access can be a powerful incentive for responsible access.

    \stopbreaks
    \emph{Good Samaritan Rule}

    First, we ask that EBMUD amend section 19 of its water service regulations, “Use and Resale of Water,” to add a Good Samaritan exception for providing water to unhoused residents of the district without premises or water service of their own.
    \startbreaks

    The second paragraph of section 19 currently reads:

    \begin{quote}
    The customer shall not permit the use of any of the water received by him from the District on any premises other than those specified in his application for service.
    \end{quote}

    We can read this paragraph to prohibit any sharing of water with homeless people, even those without premises or water service of their own. It is very broad, and for good reason.

    EBMUD could introduce a narrowly tailored exception to that broad rule for the essential, lifesaving purpose of charitable relief by amending the second paragraph of section 19 like so:

    \begin{quote}
      The customer shall not permit the use of any of the water received by him from the District on any premises other than those specified in his application for service\ul{, with the sole exception that a customer may provide up to 1,000 gallons of water received by him from the District in any single calendar month for use by people who meet the definition of “homeless” under the federal McKinney-Vento Homeless Assistance Act (42 U.S.C. Sec. 11301 et seq.) and who live in the District}.
    \end{quote}

    The McKinney-Vento Act definition is incorporated into California law for various laws relieving those in need, such as in the Vehicle Code to specify eligibility for no-cost identification cards. The rule requiring a meter per premises, as well as the rule against reselling water, in other paragraphs of section 19, would remain unchanged. The rule against resale would apply to the new Good Samaritan exception for sharing with homeless people.

    We believe this change would benefit all involved.

    First and foremost, the change would make more clean, safe drinking water available to homeless residents of the district. They desperately need it.

    Second, the change would assure Good Samaritans that providing water to homeless people in need does not break the rules of their water service, enabling more to step forward and get involved. We have seen firsthand how water can connect the housed to the unhoused in their communities.

    Third, the change would address concerns of sympathetic businesses who might be willing to provide water access to nearby camp communities. A surprising number of camps maintain good, constructive relationships with their commercial neighbors.

    Finally, the change would increase use of refillable water containers, such as jerrycans, reducing bottle trash, resultant hazards, and friction with neighbors. It would also reduce the need and temptation to appropriate, tamper with, and potentially damage EBMUD infrastructure such as fire hydrants and water lines.

    At a fundamental level, this change would show that EBMUD sees, knows and cares, by recognizing that many residents of the district—too many—don’t have any premises at which to get EBMUD service, but nonetheless share the universal need of all residents of the district for access to clean water. It is not, however, a complete path to long-term sustainability or sufficiency.

    \stopbreaks
    \emph{Hydrant Access Program}

    Second, we ask that EBMUD create a special extension to its hydrant meter program enabling local nonprofits, and camps organized as nonprofits, to access fire hydrants to provide drinking water in a responsible, accountable, affordable way.
    \startbreaks

    We understand that EBMUD operates a hydrant meter program under which commercial water users, often operators of trucks with large water tanks, borrow specialized attachments from EBMUD. They connect that equipment to fire hydrants, fill their tanks, track their usage, and self-report to EBMUD on a monthly basis. The terms for this service explicitly prohibit use of the water for drinking. The equipment is specialized to large, expensive, high-flow, three-inch-diameter hoses.

    The current hydrant meter program is well tailored to the needs of commercial users. Those users require volumes of water orders of magnitude greater than even the largest encampment. The program gives these users a responsible, accountable, and convenient way to access the water system for their particular needs. It discourages provision of water by illicit means, and without proper equipment training, by providing legitimate access, equipping with the right tools, and communicating responsible operating standards.

    Looking to other local water utilities, we also see hydrant meter options tailored to less demanding commercial users. For example, both SFPUC and San Jose Water offer ⅝-inch meters, in addition to three-inch meters. These smaller meters better suit landscaping, power washing, pest control, and special event users, whose volumes, rates, and equipment aren’t so demanding. Utilities offering this option issue smaller meters on the same deposit basis as larger meters, but at much lesser cost.

    We recommend that EBMUD establish a new program, an extension to its hydrant meter program, enabling local, nonprofit, tax-exempt charities, including camps organized as 501(c)(3) entities, to use smaller-diameter equipment to access hydrants, exclusively for provision of water to homeless residents of the district. The language of our proposed change to the water service regulations, above, might be useful again here.

    As with ⅝-inch hydrant meters for other systems, program equipment would end with male garden hose thread fittings, like household hose bibs. If necessary, the equipment could also include backflow preventers, as ⅝-inch hydrant meters from other utilities sometimes do.

    Crucially, a garden hose connection would readily mate to cartridge and disposable filters, such as those mass marketed for marine and recreational vehicles. We recommend that EBMUD require use of a filter, rated under NSF/ANSI 42 and 53 for aesthetic and health effects, as well as testing of water leaving the filter. We also recommend that the program require operation by named and trained individuals on the account, as well as temporary rather than constant connection to hydrants, to ensure good water going out of the system and nothing new or surprising going back in.

    Both EBMUD and participating nonprofits can reduce time and monetary cost of the program by further tailoring to the particular use case. For example, we expect participants in this program would nearly always fill standardized containers, such as five-gallon jerrycans and other volume-graduated vessels, on a weekly or twice-weekly basis. Thus, unlike landscapers, construction firms, or event hosts, program participants would be well positioned to self-report usage by volume, obviating the need for fragile and expensive inline flow meters.

    EBMUD could also keep costs down by leaving it to program participants to provide the filters. After all, filters are standardized, consumer-grade, consumable items. Fortunately, both disposable and cartridge varieties are widely available from a variety of manufacturers and retailers. Requiring EBMUD to procure, issue, manage, and replace these interchangeable, standardized, consumer-grade components, which do not pose the interoperability, engineering, or safety risks of the hydrant attachment itself, would only increase cost and complicate administration on EBMUD’s end.

    As it happens, a number of homeless people and nonprofit personnel in our area have practical experience in construction, firefighting, and even water utility work. Others are eager to receive training. With the right tools and skills, volunteers could help some of our most isolated camps access safe drinking water where commercial and residential Good Samaritans, much less convenience stores, aren’t available.

    When fire department and EBMUD personnel come upon the hydrants used under the program, they will find them in better condition for having been exercised recently and properly, not damaged goods. Where hydrants have been damaged already, by unregulated use or otherwise, EBMUD and program participants will share a common interest in identification, repair, protection, and responsible use going forward.

    We would be happy to recommend specific organizations as trustworthy pilot participants. We know they would jump at the chance. For many of our most isolated camps, this could represent a truly sustainable, reliable, self-sufficient solution to dire drinking water needs, needs that have gone painfully unmet for too long, and for too many.

    \stopbreaks
    \textbf{Commitment}

    In presenting these problems and opportunities to you, we also commit to make ourselves available to answer questions, clarify our proposals, and speak with you and your staff more generally. We have done our best to propose measures that we think will work for EBMUD, and not just for homeless people in need. But we know this is the beginning, not the end, of a conversation, a conversation leading to urgent action, with an openness to learning and refinement.
    \startbreaks

    We sincerely believe that instead of experiencing our housing crisis as a drain and a liability, EBMUD can be an active, visible, and meaningful contributor to innovative solutions. Looking around the Bay Area, the state, and country, we know that we are not the first to face this problem. We could be the first to demonstrate a solution fully involving the utility, community organizations, and those in need.

    \begin{samepage}
    Signed,

    Kyle Mitchell, Attorney, Oakland, kyle@kemitchell.com

    Derrick Soo, 77th Avenue Rangers, Oakland, sooderrick@yahoo.com

    Milan Griffes, Entrepreneur, Oakland, milan.griffes@gmail.com

    Berkeley Free Clinic, Berkeley, info@berkeleyfreeclinic.org

    Adobe Services, Fremont, info@abodeservices.org

    Alejandra Ruiz, Oakland

    Sergio Ruiz, Oakland

    Agustin Chavez, Oakland

    East Oakland Collective, Oakland, info@eastoaklandcollective.com
    \end{samepage}
\end{letter}
\end{document}
